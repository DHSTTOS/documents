%% LaTeX file for Design representation

%% design.tex
%% 
%% Karlsruhe Institute of Technology
%% Version 1.0, 2018-12-13

%% Available page modes: oneside, twoside
%% Available languages: english, ngerman
%% Available modes: draft, final (see README)
\documentclass[oneside, english, final]{design}

\usepackage{graphicx}
\usepackage{caption}
\usepackage{pdfpages}
\usepackage[export]{adjustbox}
\usepackage{hyperref}

\hypersetup{hidelinks,
backref=true,
pagebackref=true,
hyperindex=true,
breaklinks=true,
colorlinks=true, linkcolor=black,
urlcolor=blue,
bookmarks=true,
bookmarksopen=false,
pdftitle={Title},
pdfauthor={Author}}

%usepackage{lipsum}


%% ---------------------------------
%% | Information about the thesis  |
%% ---------------------------------

%% Name of the author
\author{PSE Group}

%% Title (and possibly subtitle) of the thesis
\title{Real-time visualization of analyzed industrial communication network traffic\\ \hfill \\Testing Report}

%% Type of the thesis 
%\thesistype{PSE}

%% The advisors are PhD Students or Postdocs
\advisor{M.Sc. Ankush Meshram}
%\begin{document}


%\end{document}
\thispagestyle{empty}

\settitle

%% --------------------------------
%% | Settings for word separation |
%% --------------------------------

%% Describe separation hints here.
%% For more details, see 
%% http://en.wikibooks.org/wiki/LaTeX/Text_Formatting#Hyphenation
\hyphenation{
% me-ta-mo-del
}

%% --------------------------------
%% | Bibliography                 |
%% --------------------------------

%% Use biber instead of BibTeX, see README
\usepackage[citestyle=numeric,style=numeric,backend=biber]{biblatex}
\usepackage{microtype}

\addtolength{\belowcaptionskip}{-10pt}
\setlength{\textfloatsep}{10pt plus 1.0pt minus 2.0pt}
%\addtolength{\abovecaptionskip}{-100pt}
\frenchspacing
%% ====================================
%% ====================================
%% ||                                ||
%% || Beginning of the main document ||
%% ||                                ||
%% ====================================
%% ====================================
\begin{document}
\nocite{*}

%% Set PDF metadata
\setpdf

%% Set the title
\maketitle

%% ----------------
%% |   Abstract   |
%% ----------------

\hfill

\begin{center}
      \large{Version 1.0.0}
\end{center}


%% The text is included from the following files:
%% - sections/abstract
\thispagestyle{empty}
\begin{abstract}
      \thispagestyle{empty}
\end{abstract}

%% -----------------
%% |   Main part   |
%% -----------------
\thispagestyle{empty}
\newpage
\thispagestyle{empty}
\tableofcontents
\cleardoublepage
\setcounter{page}{1}


\section{Testing}\label{sec:intro}
\subsection{Introduction}

This testing report covers changes and bugfixes from the one described in the implementation phase
and describes the current state of the project.

\subsection{Issues Resolved}

\subsubsection{Tooltip on Node-Link diagram not working}

See GitHub issue \href{https://github.com/DHSTTOS/implementation/issues/69}{\color{blue}{\#69}}.

\begin{itemize}
      \item{Symptom
            \\
            Tooltips don't show up somehow.}

      \item{Solution
            \\
            Heavily modified tooltips code, and applied correct CSS styles to correctly render the tooltips.}
\end{itemize}

\subsubsection{Data Store updates break everything}

Also see GitHub issue \href{https://github.com/DHSTTOS/implementation/issues/69}{\color{blue}{\#69}}.

\begin{itemize}
      \item{Symptom
            \\
            Calling myNetwork.updateData within the callback passed to MobX autorun function crashed everything.}

      \item{Solution
            \\
            The reason was that hidden deep inside the update() function of the nodelink diagram, there were some "functions" that mutate the passed in parameters themselves, adding in D3 positioning info and modifying other stuff - so in our case it's the data object inside dataStore gets mutated and it becomes useless when it gets passed inside myNetwork.updateData(), which is why it all chokes when MobX runs myNetwork.updateData() later on.
            }
\end{itemize}

\subsubsection{Incompatibility issues of data pulling}

See discussions under GitHub issue \href{https://github.com/DHSTTOS/implementation/issues/69}{\color{blue}{\#69}} and pull request \href{https://github.com/DHSTTOS/implementation/issues/78}{\color{blue}{\#78}}.

\begin{itemize}
      \item{Symptom
            \\
            Due to some issues, the frontend-backend communication was not working properly. Data pulling was not working.}

      \item{Solution
            \\
            Fixed the incompatible API calles on both server and client side.
            }
\end{itemize}

\subsubsection{Filter and grouping features not working}

See discussions under GitHub issue \href{https://github.com/DHSTTOS/implementation/issues/86}{\color{blue}{\#86}}, \href{https://github.com/DHSTTOS/implementation/issues/86}{\color{blue}{\#74}} and pull request \href{https://github.com/DHSTTOS/implementation/issues/101}{\color{blue}{\#101}}.

\begin{itemize}
      \item{Symptom
            \\
            The filter and grouping features were not working due to an incomplete implementation.}

      \item{Solution
            \\
            Completed the feature.
            }
\end{itemize}

\subsection{Remaining Issues}
\begin{itemize}
      \item{Handle logged-out state better}
      
      
      %% Some of the changes we made were :
      %% 1. Changed the layout of the node-link diagram to an ellipsoid shape for better visability
      %% 2. Added the legend alongside the diagrams, since it was missing
      %% 3. Added a mouseover function to the links. Now when a mouse hovers a link , the respective link will get highlighted and we also added som
      %% tooltip to the link that on mouseover it displays information for the source-node as well as the target-node.
      %% 4. Seperated the layers by colors as well as seperated different protocols within a layer
      %%
      %%
      %%This are the changes that were made since the implementation phase.

\end{itemize}

    Fix concurrentException in logout().

        Fix footer checkbox
	    Adjust container width, remove pink background, adjust brush height.

	        Update tickmarks when brushing.
   fix: getCollectionGroup takes collection name as parameter.

    Fix: client sessions clobbering each other: move initialization of the static maps to a static block.

        Add getRecord (GET\_RECORD) to fetch exactly matching records.


	    Add getCollectionGroupEndpoints() and DATAGROUP_ENDPOINTS. This fetch�~@#

	        In Hub, catch and rethrow all exceptions occuring during login.
    In MongoDBUserSession catch MongoSocketOpenException (e.g. for timeout due t







\end{document}
