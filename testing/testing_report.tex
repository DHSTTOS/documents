%% LaTeX file for Design representation

%% design.tex
%% 
%% Karlsruhe Institute of Technology
%% Version 1.0, 2018-12-13

%% Available page modes: oneside, twoside
%% Available languages: english, ngerman
%% Available modes: draft, final (see README)
\documentclass[oneside, english, final]{design}

\usepackage{graphicx}
\usepackage{caption}
\usepackage{pdfpages}
\usepackage[export]{adjustbox}
\usepackage{hyperref}

\hypersetup{hidelinks,
backref=true,
pagebackref=true,
hyperindex=true,
breaklinks=true,
colorlinks=true, linkcolor=black,
urlcolor=blue,
bookmarks=true,
bookmarksopen=false,
pdftitle={Title},
pdfauthor={Author}}

%usepackage{lipsum}


%% ---------------------------------
%% | Information about the thesis  |
%% ---------------------------------

%% Name of the author
\author{PSE Group}

%% Title (and possibly subtitle) of the thesis
\title{Real-time visualization of analyzed industrial communication network traffic\\ \hfill \\Testing Report}

%% Type of the thesis 
%\thesistype{PSE}

%% The advisors are PhD Students or Postdocs
\advisor{M.Sc. Ankush Meshram}
%\begin{document}


%\end{document}
\thispagestyle{empty}

\settitle

%% --------------------------------
%% | Settings for word separation |
%% --------------------------------

%% Describe separation hints here.
%% For more details, see 
%% http://en.wikibooks.org/wiki/LaTeX/Text_Formatting#Hyphenation
\hyphenation{
% me-ta-mo-del
}

%% --------------------------------
%% | Bibliography                 |
%% --------------------------------

%% Use biber instead of BibTeX, see README
\usepackage[citestyle=numeric,style=numeric,backend=biber]{biblatex}
\usepackage{microtype}

\addtolength{\belowcaptionskip}{-10pt}
\setlength{\textfloatsep}{10pt plus 1.0pt minus 2.0pt}
%\addtolength{\abovecaptionskip}{-100pt}
\frenchspacing
%% ====================================
%% ====================================
%% ||                                ||
%% || Beginning of the main document ||
%% ||                                ||
%% ====================================
%% ====================================
\begin{document}
\nocite{*}

%% Set PDF metadata
\setpdf

%% Set the title
\maketitle

%% ----------------
%% |   Abstract   |
%% ----------------

\hfill

\begin{center}
      \large{Version 1.0.0}
\end{center}


%% The text is included from the following files:
%% - sections/abstract
\thispagestyle{empty}
\begin{abstract}
      \thispagestyle{empty}
\end{abstract}

%% -----------------
%% |   Main part   |
%% -----------------
\thispagestyle{empty}
\newpage
\thispagestyle{empty}
\tableofcontents
\cleardoublepage
\setcounter{page}{1}


\section{Testing}\label{sec:intro}
\subsection{Introduction}

This testing report covers changes and bugfixes from the one described in the implementation phase
and describes the current state of the project.

\subsection{Issues Resolved}

\subsubsection{Tooltip on Node-Link diagram not working}

See GitHub issue \href{https://github.com/DHSTTOS/implementation/issues/69}{\color{blue}{\#69}}.

\begin{itemize}
      \item{Symptom
            \\
            Tooltips don't show up somehow.}

      \item{Solution
            \\
            Heavily modified tooltips code, and applied correct CSS styles to correctly render the tooltips.}
\end{itemize}

\subsubsection{Data Store updates break everything}

Also see GitHub issue \href{https://github.com/DHSTTOS/implementation/issues/69}{\color{blue}{\#69}}.

\begin{itemize}
      \item{Symptom
            \\
            Calling myNetwork.updateData within the callback passed to MobX autorun function crashed everything.}

      \item{Solution
            \\
            The reason was that hidden deep inside the update() function of the nodelink diagram, there were some "functions" that mutate the passed in parameters themselves, adding in D3 positioning info and modifying other stuff - so in our case it's the data object inside dataStore gets mutated and it becomes useless when it gets passed inside myNetwork.updateData(), which is why it all chokes when MobX runs myNetwork.updateData() later on.
            }
\end{itemize}

\subsubsection{Incompatibility issues of data pulling}

See discussions under GitHub issue \href{https://github.com/DHSTTOS/implementation/issues/69}{\color{blue}{\#69}} and pull request \href{https://github.com/DHSTTOS/implementation/issues/78}{\color{blue}{\#78}}.

\begin{itemize}
      \item{Symptom
            \\
            Due to some issues, the frontend-backend communication was not working properly. Data pulling was not working.}

      \item{Solution
            \\
            Fixed the incompatible API calles on both server and client side.
            }
\end{itemize}

\subsubsection{Realtime data being persistent}

When listening for Realtime data there was not limit to how much of a buffer of previous sessions.

\begin{itemize}
      \item{Symptom
            \\
            Closing and restarting the program at a latter time would keep the real-time collection when the two distinct times should not be concatenated together on the same collection.
	}
      \item{Solution
            \\
            When the program first starts, discard all real-time date pertaining to other sessions and creating a buffer from there.
            }
\end{itemize}


\subsubsection{Realtime data buffer keeps growing}

when listening for Realtime data there was not limit to how much of a buffer of the past 

\begin{itemize}
      \item{Symptom
            \\
            Once we started listening to real-time there was no limit to how big the buffer grew. As it grew queries become slower and more unnecessary data was stored.
	}
      \item{Solution
            \\
            Limit the size of the real-time collection to 60K entries. Roughly 10 minutes of buffer at 100 entries per second.
            }
\end{itemize}

\subsubsection{Real-time  aggregated collections not being updated}

Real-time aggregations were not being updated since the java API does not overwrite entries on Mongo by default.

\begin{itemize}
      \item{Symptom
            \\
            Real-time aggregated collections are not updated.
	}
      \item{Solution
            \\
            Iterate over all aggregated collections and delete them on start alongside the real-time collection.
            }
\end{itemize}

\subsubsection{Real-time  aggregated collections not visible from front end}

\begin{itemize}
      \item{Symptom
            \\
            Real-time collections and aggregations do not show on the front-end. Since they do not exist if the program is not listening to the real-time Kafka topic.
	}
      \item{Solution
            \\
            On start, if not existing, create empty collections for real-time and it's aggregations to expose them to the front end.
            }
\end{itemize}

\subsubsection{Getting collections records between a range}



\begin{itemize}
      \item{Symptom
            \\
            The range was inclusive at the start and inclusive at the end when it should be inclusive at start and exclusive at the end.
	}
      \item{Solution
            \\
            Changing the query to Mongo from `less or equal than' to `less than'.
            }
\end{itemize}

\subsubsection{Getting collections records between a range when using \_id }

\begin{itemize}
      \item{Symptom
            \\
            The \_id corresponds to the index of the collection. Getting for example the records from 0 as start to end 10 does not return records 0-9
	}
      \item{Solution
            \\
            Changed \_id from a String to a long in the collection to allow for numerical comparisons.
            }
\end{itemize}



\subsubsection{collections are too large to effectively draw the node link diagram}

with really big datasets it's problematic to iterate over the whole thing to draw the diagram.

\begin{itemize}
      \item{Symptom
            \\
            Hang ups and freezes until the diagram is drawn and continuous freezes on redraws.
	}
      \item{Solution
            \\
            Created a new aggregation containing the information needed to draw the diagram including all connections over a data set as well as the ability to create subsets spanning a set range.
            }
\end{itemize}

\subsubsection{Handling of dates from Kafka and Mongo}

When converting an entry from Kafka to json containing a date the value it is converted to {\$date : long} but bson doesn't automatically convert this to a Date Java object.

\begin{itemize}
      \item{Symptom
            \\
            Automatic failure of the bson parsing routine
	}
      \item{Solution
            \\
            Several revisions of this.
            \\
            First fix was ignoring the \$date object altogether and create a mock Timestamp class holding the real value. while this fixes the issue it introduces another one where getting an entry from the MongoDB and using bson to convert it into a Java object does not work since the conversion does not hold this mock Timestamp class.\\
            Second fix was writing a deserialization routine and creating a two classes, one containing the mock object and one without it. Fixes the issue but under testing it was found unreliable and requiring way too big of an overhead.\\
            Final fix: Create a proper deserialization routine converting the portion of the record containing the date into it's own json object and extracting the value of \$date as key fixes the issue when converting from Kafka. Converting from Mongo skips the routine and works as intended
            }
\end{itemize}


\subsubsection{Filter and grouping features not working}

See discussions under GitHub issue \href{https://github.com/DHSTTOS/implementation/issues/86}{\color{blue}{\#86}}, \href{https://github.com/DHSTTOS/implementation/issues/86}{\color{blue}{\#74}} and pull request \href{https://github.com/DHSTTOS/implementation/issues/101}{\color{blue}{\#101}}.

\begin{itemize}
      \item{Symptom
            \\
            The filter and grouping features were not working due to an incomplete implementation.}

      \item{Solution
            \\
            Completed the feature.
            }
\end{itemize}


\subsection{Remaining Issues}
\begin{itemize}
      \item{Handle logged-out state better}
      
      
      %% Some of the changes we made were :
      %% 1. Changed the layout of the node-link diagram to an ellipsoid shape for better visability
      %% 2. Added the legend alongside the diagrams, since it was missing
      %% 3. Added a mouseover function to the links. Now when a mouse hovers a link , the respective link will get highlighted and we also added som
      %% tooltip to the link that on mouseover it displays information for the source-node as well as the target-node.
      %% 4. Seperated the layers by colors as well as seperated different protocols within a layer
      %%
      %%
      %%This are the changes that were made since the implementation phase.

\end{itemize}


\end{document}



