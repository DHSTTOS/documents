\documentclass[oneside, english, final]{design}

\usepackage{graphicx}
\usepackage{caption}
\usepackage{pdfpages}
\usepackage[export]{adjustbox}
\usepackage{hyperref}

\begin{document}

{\large Deployment/Installation Manual for the ADIN Inspector System}

\section{On the backend server}
\subsection{Kafka}
\subsection{MongoDB}

\subsection{Apache Tomcat}
\subsubsection{tomcat configuration}
May vary depending on the specific underlying OS.
On digital ocean's droplet hooking tomcat up to the systemd demon resulted in the server aborting and restarting in a loop; starting tomcat manually (with "catalina.sh start" as tomcat user) did work.
\\
Note: Tomcat needs random numbers on start (reads /dev/random or so) for ssl stuff; if the server it's running on is low on entropy (e.g. because it sees little network traffic and other activity) this can cause a delay before Tomcat actually starts to handle network requests, of up to 3 minutes.


\subsubsection{the webapps/adininsspector content}
\begin{itemize}
\item{copy the .jar (also the gson.jar?) to webapps/adininspector/WEB-INF/libs/}
\item{optional, for testing: copy and make readable the websockclient2.html file.}
\end{itemize}

\subsection{Firewall}
If a firewall is installed, allow ports
\begin{itemize}
\item{80, 443 for production} 
\item{3000 (optional, for testing with npm-based webserver)}
\item{8080, 8433 needed in the development environmnet, not for production}
\end{itemize}

\section{On the frontend server}
\subsection{Webserver}
\begin{itemize}
\item{install and configure the webserver}
\item{set up the directory with the web pages and javascript stuff}
\end{itemize}

\subsection{Firewall}
If a firewall is installed, allow ports
\begin{itemize}
\item{80, 443 for production} 
\item{3000 (optional, for testing with npm-based webserver)}
\item{8080, 8433 needed in the development environmnet, not for production}
\end{itemize}
\end{document}
