%% LaTeX file for Design representation
%% design.tex
%% 
%% Karlsruhe Institute of Technology
%% Version 1.0, 2018-12-13

%% Available page modes: oneside, twoside
%% Available languages: english, ngerman
%% Available modes: draft, final (see README)
\documentclass[oneside, english, final]{design}

\usepackage{graphicx}
\usepackage{caption}
\usepackage{pdfpages}
\usepackage[export]{adjustbox}
\usepackage{hyperref}

\hypersetup{hidelinks,
backref=true,
pagebackref=true,
hyperindex=true,
breaklinks=true,
colorlinks=true, linkcolor=black,
urlcolor=blue,
bookmarks=true,
bookmarksopen=false,
pdftitle={Title},
pdfauthor={Author}}

%usepackage{lipsum}


%% ---------------------------------
%% | Information about the thesis  |
%% ---------------------------------

%% Name of the author
\author{PSE Group}

%% Title (and possibly subtitle) of the thesis
\title{Real-time visualization of analyzed industrial communication network traffic\\ \hfill \\Implementation Report}

%% Type of the thesis 
%\thesistype{PSE}

%% The advisors are PhD Students or Postdocs
\advisor{M.Sc. Ankush Meshram}
%\begin{document}


%\end{document}
\thispagestyle{empty}

\settitle

%% --------------------------------
%% | Settings for word separation |
%% --------------------------------

%% Describe separation hints here.
%% For more details, see 
%% http://en.wikibooks.org/wiki/LaTeX/Text_Formatting#Hyphenation
\hyphenation{
% me-ta-mo-del
}

%% --------------------------------
%% | Bibliography                 |
%% --------------------------------

%% Use biber instead of BibTeX, see README
\usepackage[citestyle=numeric,style=numeric,backend=biber]{biblatex}
\usepackage{microtype}

\addtolength{\belowcaptionskip}{-10pt}
\setlength{\textfloatsep}{10pt plus 1.0pt minus 2.0pt}
%\addtolength{\abovecaptionskip}{-100pt}
\frenchspacing
%% ====================================
%% ====================================
%% ||                                ||
%% || Beginning of the main document ||
%% ||                                ||
%% ====================================
%% ====================================
\begin{document}
\nocite{*}

%% Set PDF metadata
\setpdf

%% Set the title
\maketitle

%% ----------------
%% |   Abstract   |
%% ----------------

\hfill

\begin{center}
	\large{Version 1.0.0}
\end{center}


%% The text is included from the following files:
%% - sections/abstract
\thispagestyle{empty}
\begin{abstract}
	\thispagestyle{empty}
\end{abstract}

%% -----------------
%% |   Main part   |
%% -----------------
\thispagestyle{empty}
\newpage
\thispagestyle{empty}
\tableofcontents
\cleardoublepage
\setcounter{page}{1}


\section{Design}\label{sec:intro}
\subsection{Introduction}
\subsection{Changes in the Design}
from tipps.pdf 7.2, page 15:
"Dokumentation "uber "Anderungen am Entwurf, beispielsweise entfernte oder neu hinzugef"ugte 
Klassen und Methoden. Gruppiert (und zusammengefasst) werden sollte nach dem Grund f"ur
die "Anderung und nicht nach der ge"anderten Klasse."

\subsection{List of implemented must- and should-criteria}
\subsubsection{List of implemented must-criteria}
FR100, FR110, FR200, FR300, FR400, FR700, FR710,
\\
not yet: FR500, FR800, FR900, FR910 \\
FR1300, FR1310 filtering\\
ask leo: FR720

\subsubsection{List of implemented should-criteria}
\begin{itemize}
  \item{FR1332 filter to compute flow rate}
  \begin{itemize}
    \item{this has instead been implemented in the backend which provides this as a new data stream}
  \end{itemize}
  \item{FR1400}
\end{itemize}

\subsubsection{List of not implemented must-criteria}
\begin{itemize}
  \item{FR600 dynamically change the selected/displayed components}
  \item{FR1000 auto scroll}
  \item{FR1100 pick data points, hover}
  \item{FR1110 node-link diagram: picking both nodes and links}
  \item{FR1200 selecting data points}
  \item{FR1210 create new diagram from selected data}
  \item{FR1330}
\end{itemize}


\subsubsection{List of not implemented should-criteria}
\begin{itemize}
\item{FR1320 per-diagram filters}
\end{itemize}


\subsection{Delays}
Welche Verz"ogerungen gab es im Implementierungsplan? Kann beispielsweise als zweites
GANTT Diagramm am Ende dargestellt werden.
\subsection{Overview of unit tests}
\end{document}
