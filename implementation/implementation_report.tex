%% LaTeX file for Design representation
%% design.tex
%% 
%% Karlsruhe Institute of Technology
%% Version 1.0, 2018-12-13

%% Available page modes: oneside, twoside
%% Available languages: english, ngerman
%% Available modes: draft, final (see README)
\documentclass[oneside, english, final]{design}

\usepackage{graphicx}
\usepackage{caption}
\usepackage{pdfpages}
\usepackage[export]{adjustbox}
\usepackage{hyperref}

\hypersetup{hidelinks,
backref=true,
pagebackref=true,
hyperindex=true,
breaklinks=true,
colorlinks=true, linkcolor=black,
urlcolor=blue,
bookmarks=true,
bookmarksopen=false,
pdftitle={Title},
pdfauthor={Author}}

%usepackage{lipsum}


%% ---------------------------------
%% | Information about the thesis  |
%% ---------------------------------

%% Name of the author
\author{PSE Group}

%% Title (and possibly subtitle) of the thesis
\title{Real-time visualization of analyzed industrial communication network traffic\\ \hfill \\Implementation Report}

%% Type of the thesis 
%\thesistype{PSE}

%% The advisors are PhD Students or Postdocs
\advisor{M.Sc. Ankush Meshram}
%\begin{document}


%\end{document}
\thispagestyle{empty}

\settitle

%% --------------------------------
%% | Settings for word separation |
%% --------------------------------

%% Describe separation hints here.
%% For more details, see 
%% http://en.wikibooks.org/wiki/LaTeX/Text_Formatting#Hyphenation
\hyphenation{
% me-ta-mo-del
}

%% --------------------------------
%% | Bibliography                 |
%% --------------------------------

%% Use biber instead of BibTeX, see README
\usepackage[citestyle=numeric,style=numeric,backend=biber]{biblatex}
\usepackage{microtype}

\addtolength{\belowcaptionskip}{-10pt}
\setlength{\textfloatsep}{10pt plus 1.0pt minus 2.0pt}
%\addtolength{\abovecaptionskip}{-100pt}
\frenchspacing
%% ====================================
%% ====================================
%% ||                                ||
%% || Beginning of the main document ||
%% ||                                ||
%% ====================================
%% ====================================
\begin{document}
\nocite{*}

%% Set PDF metadata
\setpdf

%% Set the title
\maketitle

%% ----------------
%% |   Abstract   |
%% ----------------

\hfill

\begin{center}
	\large{Version 1.0.0}
\end{center}


%% The text is included from the following files:
%% - sections/abstract
\thispagestyle{empty}
\begin{abstract}
	\thispagestyle{empty}
\end{abstract}

%% -----------------
%% |   Main part   |
%% -----------------
\thispagestyle{empty}
\newpage
\thispagestyle{empty}
\tableofcontents
\cleardoublepage
\setcounter{page}{1}


\section{Design}\label{sec:intro}
\subsection{Introduction}
XXX
\\
\subsection{Changes in the Design}
XXX
\\
from tipps.pdf 7.2, page 15:
"Dokumentation "uber "Anderungen am Entwurf, beispielsweise entfernte oder neu hinzugef"ugte 
Klassen und Methoden. Gruppiert (und zusammengefasst) werden sollte nach dem Grund f"ur
die "Anderung und nicht nach der ge"anderten Klasse."


\subsubsection{Refactoring for cleaner code and changes for convenience reasons}
\begin{itemize}
\item{Add parameter
  \\
	Added parameter DBname to \textsf{\textbf{MongoConsumer(user, pass, dbName)}} for creating a reference to pass onto the MongoClientMediator}

\item{Refactoring
  \\
  Add attribute \textsf{\textbf{private KafkaConsumer<String, String> consumer}} because other functions need to use the consumer}

  \item{Refactor: extract instance attribute}
    \\ 
    Add attribute \textsf{\textbf{private MongoDatabase db}}
    as a reference to the database all methods need to access.

  \item{Convenience functions for different data types}
    \\
    Added variations of \textsf{\textbf{addRecordToCollection(Record record, String collection)}}
    that take a document or an list of documents or an array of record sinstead of a Record.


  \item{Add convenience function}
    \\
    Added \textsf{\textbf{getCollectionAsRecordsArrayList()}}
    to DataProcessor.

  \item{Refactor passing the current mediator object}
  \\
    Add parameter \textsf{\textbf{MongoClientMediator}} to \textsf{\textbf{public static void ProcessData:processData(String collectionName, MongoClientMediator clientMediator)}} so that \textsf{\textbf{processData}} can use it to write the processed data to the database. Remove attribute \textsf{\textbf{ProcessData:MongoClientMediator client}} which was used for this before.


  \item{Add convenience function}
    \\
    Add method \textsf{\textbf{public static void processData(ArrayList<String> collectionNames, MongoClientMediator clientMediator)}} to process a list of collections (instead of calling \textsf{\textbf{processData}} for each collection.
    
  \item{Add convenience function}
    \\
    Added method \textsf{\textbf{public Document getNewAggregatorDocument(Date tstmp)}} for easier handling of date values.

    
  \item{Add convenience attributes}
    \\
    Add the variables Variables
    \textsf{\textbf{private ArrayList<Map<String, Object>> connectionsMapList}} and
    \textsf{\textbf{private Document currentDocument}} to the classes \textsf{\textbf{FlowRatePerSecond}} and \textsf{\textbf{NumberOfConnectionsPerNode}} to
    keep track of which document is being processed now and which connections happened within this second.


\item{Refactoring for cleaner code in protocol handling}
  \\
  Change the protocol parsing in class \textsf{\textbf{ClientProtocoHandler}} from a switch construct to using a private enum.

\end{itemize}

\subsubsection{Changes because of clarified requirements}
\begin{itemize}
\item{Differing input formats for Date/Timestamp}
  \\
  Split class \textsf{\textbf{PacketRecord}} into \textsf{\textbf{PacketRecordDesFromMongo}} and \textsf{\textbf{PacketRecordDesFromKafka}} to handle different formats.

\end{itemize}


\subsubsection{Changes because of oversights}

\begin{itemize}
  \item{added dbName to MongoClientMediator since we need to know from which DB we want to read/write collections.}
\\


  \item{Unspecified return type}
    \\
    The return type of \textsf{\textbf{public ArrayList<Document> processData( ArrayList<Record> records)}} in \textsf{\textbf{IAggregator}}
was unspecified in the Design document.

  \item{Session handling}
    \\
    To handle session state, \textsf{\textbf{Hub:login()}}, \textsf{\textbf{Hub:loginWithToken()}}, and \textsf{\textbf{Hub:logout()}} were added.
    To keep track of client session state, the private attributes \textsf{\textbf{Hub:sessions}} and \textsf{\textbf{Hub:loginTokens}} were added.

\end{itemize}


\subsubsection{Changes because of unexpected complexity}
\begin{itemize}
  \item{Workaround for Kafka's API}
    \\
    Change \textsf{\textbf{getAllTopics()}} to \textsf{\textbf{getAllTopicsPartitions()}}:
    return a Collection of topic partitions essentially to force kafka to send all records from the start.
    It was complex to make kafka read all the topics from the beginning.
    Secondary aspect: convenient because it relegates topic creation to another method.

  \item{Workaround for Kafka's API}
    \\
    Add method \textsf{\textbf{ArrayList<String> getTopicsForProcessing()}}
  because there are some topics in kakfka which are for internal use, e.g. \_\_consumeroffsets.
 This returns the topics we need to process.
	
  \item{Exception handling}
    \\ 
    The constructor for class \textsf{\textbf{MongoClientMediator}} now throws a LoginFailureException instead of forwarding an unchecked exception.

  \item{Converting between different APIs}
    \\ 
    Add method \textsf{\textbf{mongoIteratorToStringArray(MongoIterable)}}
    because the hub expects an array but the mongodb returns a MongoIterable.

  \item{Handling the login happening in another websocket session than the main app}
    \\
    To deal with a restart of the websocket connection when changing from the login page to the main page, session handling was changed. Added the \textsf{\textbf{LOGIN\_TOKEN}} request to the protocol and \textsf{\textbf{Hub:loginWithToken}}.

  \item{Adapt to React and MobX}
    \\
    To adapt to the observer-driven architecture of React and MobX, store data from the server in datastructures \textsf{\textbf{dataStore.rawData}} and \textsf{\textbf{dataStore.alarms}}
    instead of returning it as return values of
    \textsf{\textbf{getAvailableCollections()}},
    \textsf{\textbf{getCollection()}}, \textsf{\textbf{getCollectionSize()}}, \textsf{\textbf{getRecordsInRange()}} and \textsf{\textbf{getRecordsInRangeSize()}} in \textsf{\textbf{wsutils.js}}.

\end{itemize}


\subsection{List of implemented must- and should-criteria}
\subsubsection{List of implemented must-criteria}
FR100, FR110, FR200, FR300, FR400, FR700, FR710, FR720, FR1310
\\
in progress: FR500, FR1300
\\
not yet: FR800, FR900, FR910

\subsubsection{List of implemented should-criteria}
\begin{itemize}
  \item{FR1332 filter to compute flow rate}
  \begin{itemize}
    \item{this has instead been implemented in the backend which provides this as a new data stream}
  \end{itemize}
  \item{FR1400}
\end{itemize}

\subsubsection{List of not implemented must-criteria}
\begin{itemize}
  \item{FR600 dynamically change the selected/displayed components}
  \item{FR1000 auto scroll}
  \item{FR1100 pick data points, hover}
  \item{FR1110 node-link diagram: picking both nodes and links}
  \item{FR1200 selecting data points}
  \item{FR1210 create new diagram from selected data}
  \item{FR1330}
\end{itemize}


\subsubsection{List of not implemented should-criteria}
\begin{itemize}
\item{FR1320 per-diagram filters}
\end{itemize}


\subsection{Delays}
XXX
\\
Welche Verz"ogerungen gab es im Implementierungsplan? Kann beispielsweise als zweites
GANTT Diagramm am Ende dargestellt werden.
\subsection{Overview of unit tests}
XXX
\\
\end{document}
